\documentclass{article}
\usepackage[utf8]{inputenc}
\usepackage{amsfonts}
\usepackage{algorithm2e}
\usepackage{amsmath}
\usepackage[a4paper]{geometry}
\geometry{hscale=0.8,vscale=0.9,centering}
\usepackage{graphicx}
\usepackage{program}
\usepackage{ulem}
\usepackage{xcolor}
\usepackage{pdfpages}
\usepackage{hyperref}
\newcommand{\subsubsubsection}[1]{\paragraph{#1}\mbox{}\\}
\setcounter{secnumdepth}{4}
\setcounter{tocdepth}{4}
\newcommand{\alinea}{
\textbf{\hspace{8mm}}
}
 \setlength{\parindent}{0pt}
 
\newcommand{\sautligne}{
\textbf{\vspace{5mm}}
}
\usepackage{pdflscape}
\newenvironment{changemargin}[2]{%
\begin{list}{}{%
\setlength{\topsep}{0pt}%
\setlength{\leftmargin}{#1}%
\setlength{\rightmargin}{#2}%
\setlength{\listparindent}{\parindent}%
\setlength{\itemindent}{\parindent}%
\setlength{\parsep}{\parskip}%
}%
\item[]}{\end{list}}
\usepackage{array,multirow,makecell}
\usepackage{subcaption}


\title{M1 Info – ARC - Lecture 6}
\author{Olivier HUREAU - Groupe 3}
\date{08/04/2020}

\begin{document}
\maketitle


\section{Exercice 1}
\subsection{Step one}
Even if with a one hot encoding, the truth tables for the transition and output functions is still the same such as :
\sautligne

\begin{figure}[h!]
\begin{subfigure}{0.5\textwidth}
\centering
	\begin{tabular}{l | c | c}
	
	c & q & q' \\
\hline	
   0 & init & init \\
   0 & Sone & Sone \\
   0 & Stwo & Stwo  \\
   0 & Sthree & Sthree \\
   1 & init & Sone \\
   1 & Sone & Stwo \\
   1 & Stwo  & Sthree \\
   1 & Sthree & init \\
 \end{tabular}

 \caption{transition}
\end{subfigure}
\hfill 
\begin{subfigure}{0.5\textwidth}
\centering
  \begin{tabular}{ c | c}
	
	state q & $Z_1$ $Z_0$ \\
\hline	
    init & 00 \\
    Sone & 01 \\
   Stwo & 10  \\
    Sthree & 11 \\
   
 \end{tabular}
  \caption{output functions}
\end{subfigure}
 
 \caption{ truth tables for the transition and output functions}
 \end{figure}
 

  
 

\subsection{Step 2}

The 2-bit counters is composed of 4 states : \{Init, Sone, Stwo, Sthree\}.

As we know  : "with One hot encoding  nb\_states bits are needed". 

Then here we need 4 bits such as \{Q0 ; Q1 ; Q2 ; Q3\} for the encoding. 

Thus we have : 
\begin{itemize}
\item Init : 0001
\item Sone : 0010
\item Stwo : 0100
\item Sthree : 1000
\end{itemize}

\subsection{Step 3}


\begin{figure}[h!]
\begin{subfigure}{0.5\textwidth}
\centering
	\begin{tabular}{l | c | c}
	
	c &  $q_3$ $q_2$ $q_1$ $q_0$ &  $q'_3$ $q'_2$ $q'_1$ $q'_0$ \\
\hline	
   0 &  0001 &  0001 \\
   0 & 0010 & 0010 \\
   0 & 0100 & 0100  \\
   0 & 1000 & 1000 \\
   1 & 0001 & 0010 \\
   1 & 0010 & 0100 \\
   1 & 0100  & 1000 \\
   1 & 1000 &  0001 \\
 \end{tabular}

 \caption{transition}
\end{subfigure}
\hfill 
\begin{subfigure}{0.5\textwidth}
\centering
  \begin{tabular}{ c | c}
	
	 $q_3$ $q_2$ $q_1$ $q_0$ & $Z_1$ $Z_0$ \\
\hline	
    0001 & 00 \\
    0010 & 01 \\
   0100 & 10  \\
    1000 & 11 \\
   
 \end{tabular}
  \caption{output functions}
\end{subfigure}
 
 \caption{ truth tables for the transition and output functions with one hot encoding}
 \end{figure}
 \newpage
Boolean expressions for t and f are derived from the truth tables (and simplified): 

\begin{itemize}
\item $q'_0$ = not(c).not(q3).not(q2).not(q1).q0 +  c.q3.not(q2).not(q1).not(q0)= not(c).q0 + c.q3
\item $q'_1$ = not(c).not(q3).not(q2).q1.not(q0) +  c.not(q3).not(q2).not(q1).q0 = not(c).q1 + c.q0
\item $q'_2$ = not(c).not(q3).q2.not(q1).not(q0) + c.not(q3).not(q2).q1.not(q0) = not(c).q2 + c.q1
\item $q'_3$ = not(c).q3.not(q2).not(q1).not(q0) + c.not(q3).not(q2).not(q1).q0 = not(c).q3 + c.q0
\item $Z_0$ = not(q3).not(q2).q1.not(q0) + q3.not(q2).not(q1).not(q0) = q1 + q3
\item $Z_1$ = not(q3).q2.not(q1).not(q0) + q3.not(q2).not(q1).not(q0) = q2 + q3
\end{itemize} 


\section{Exercice 2}
Here, we have 8 states so the one hot encoding will be on 8 bits
Assume that the following binary encoding has been used: AT1a=00000001, AT1=00000010, B12=00000100, AT2=0000 1000,
AT2a=00010000, B23=00100000, AT3=01000000, AT3a=10000000

\alinea  And that the signal identifiers of the 8 resulting flips-flops are
S7, S6, S5, S4, S3, S2, S1, and S0. 
\sautligne

Then, the adaptation of the PSL assertions is:
\begin{verbatim}
always(r1 -> eventually!(S7=0 and S6=0 and S5=0 and S4=0 and S3=0 and S2=0 and S1=1 and S0=0 and))
always(r2 -> eventually!(S7=0 and S6=0 and S5=0 and S4=0 and S3=1 and S2=0 and S1=0 and S0=0 and))
always(r3 -> eventually!(S7=0 and S6=1 and S5=0 and S4=0 and S3=0 and S2=0 and S1=0 and S0=0 and))


\end{verbatim}


\section{Exercice 3}


\subsection{Bulding Truth table}

\begin{figure}[h]
\centering
\begin{tabular}{c c | c | c}
	
	A & B & q & q' \\
\hline	
	0 & 0 & init & init \\
	0 & 0 & S1 & init \\
	0 & 1 & init & init \\
	0 & 1 & S1 & S1 \\
	1 & 0 & init & init  \\
	1 & 0 & S1 & S1 \\
	1 & 1 & init & S1 \\
	1 & 1 & S1 & S1\\	
 \end{tabular}
\caption{Transition truth table}
\end{figure}

As the aumtomaton is a mealy machine, outputs depends on the state and the value of A and B. Thus we have

\begin{figure}[h]
\centering
\begin{tabular}{c c  c | c}
	
	A & B & q & $Z_0$ \\
\hline	
	0 & 0 & init & 0 \\
	0 & 0 & S1 & 1 \\
	0 & 1 & init & 1 \\
	0 & 1 & S1 & 0 \\
	1 & 0 & init & 1  \\
	1 & 0 & S1 & 0 \\
	1 & 1 & init & 0 \\
	1 & 1 & S1 & 1 \\	
 \end{tabular}
\caption{Output truth table}
\end{figure}

\subsection{Choice encoding}
As we are working with FPGA and the One hot encoding  is more appropriate in
the context of FPGA synthesis. The state encoding will be One hot. We have 2 states then 2 bits are needed.

That is : 
\begin{itemize}
\item init = 01
\item S1 = 10
\end{itemize}

\subsection{Truth table with One hot encoding} :


 \begin{figure}[h!]
\begin{subfigure}{0.5\textwidth}
\centering
\begin{tabular}{c c | c | c}
	
	A & B & $q_1$ $q_0$ & $q'_1$ $q'_0$ \\
\hline	
	0 & 0 & 01 & 01 \\
	0 & 0 & 10 & 01 \\
	0 & 1 & 01 & 01 \\
	0 & 1 & 10 & 10 \\
	1 & 0 & 01 & 01  \\
	1 & 0 & 10 & 10 \\
	1 & 1 & 01 & 10 \\
	1 & 1 & 10 & 10 \\	
 \end{tabular}

 \caption{transition}
\end{subfigure}
\hfill 
\begin{subfigure}{0.5\textwidth}
\centering
\begin{tabular}{c c  c | c}
	
	A & B & $q_1$ $q_0$ & $Z_0$ \\
\hline	
	0 & 0 & 01 & 0 \\
	0 & 0 & 10 & 1 \\
	0 & 1 & 01 & 1 \\
	0 & 1 & 10 & 0 \\
	1 & 0 & 01 & 1  \\
	1 & 0 & 10 & 0 \\
	1 & 1 & 01 & 0 \\
	1 & 1 & 10 & 1 \\	
 \end{tabular}
  \caption{output functions}
\end{subfigure}
 
 \caption{ truth tables for the transition and output functions with One hot encoding}
 
 and Boolean expressions for t and f are derived from the truth tables (and simplified):
 
 
 \begin{itemize}
 \item Q0' = 
 not(A).not(B).not(Q1).Q0 + 
 not(A).not(B).Q1.not(Q0) +
 not(A).B.not(Q1).Q0 + 
 A.not(B).not(Q1).Q0 = not(A).not(B) +   not(A).B.not(Q1).Q0 + 
 A.not(B).not(Q1).Q0
 \item Q1' : 
 not(A).B.Q1.not(Q0) + 
 A.not(B).Q1.not(Q0) +
A.B.not(Q1).not(Q0) + 
A.B.not(Q1).not(Q0) = A.B +  not(A).B.Q1.not(Q0) + 
 A.not(B).Q1.not(Q0) +
 \item Z0 : 
 not(A).not(B).Q1.not(Q0) + 
 not(A).B.not(Q1).Q0 +
 A.not(B).Q1.not(Q0) + 
A.B.Q1.not(Q0)
 \end{itemize}
 \end{figure}

\end{document}